\rhead{\emph{Schrift oben rechts}}
\section{Hauptkapitel}\label{sec:hauptkapitel}
Beispielsdokument (\emph{bzw. Vorlage}) für eine \ac{BAT}\footnote{Bei Mehrzahl kann apc statt ac verwendet werden.}. Es folgt ein Zeilenumbruch.\\ 
~\\ % leere Zeile
Ich bin ein Quellenverweise, gesehen in \cite[S.~57]{KeilStefan2017D}. Ein Text in "'Anführungszeichen"'.\\\\
\lipsum[1]
\begin{figure}[H]
	\centering
	\includegraphics[width=0.75\textwidth]{../fig/placeholder}
	\caption{Platzhalter für ein Bild}
	\label{fig:testaufbau_uebersicht}
\end{figure}
\subsection{Unterkapitel}
\lipsum[1]\\\\
Es folgt ein Seitenumbuch.
\newpage \noindent
\textbf{Nicht nummerierte Aufzählung:}
\begin{itemize}[itemsep=-5pt,topsep=0pt]
	\item Element 1 einer Aufzählung
	\item Element 2 einer Aufzählung
	\begin{enumerate}[a), itemsep=-2pt,topsep=-6pt]
		\item Unterelement von 2 
		\item Ebenso
	\end{enumerate}
	\item Element 3 einer Aufzählung
\end{itemize}
\textbf{Nummerierte Aufzählung:}
\begin{enumerate}[itemsep=-5pt,topsep=0pt]
	\item Element 1 einer Aufzählung
	\item Element 2 einer Aufzählung
	\item Element 3 einer Aufzählung
\end{enumerate}
\subsubsection{Unterunterkapitel}
Zwei Bilder nebeneinander.
\begin{figure}[H]
	\centering
	\captionsetup[subfigure]{justification=centering}
	\begin{subfigure}[b]{0.48\textwidth}
  	\centering
		\includegraphics[width=0.95\linewidth]{../fig/placeholder}
		\caption{Bild links}
		\label{fig:left}
	\end{subfigure}
	\begin{subfigure}[b]{0.48\textwidth}
		\centering
		\includegraphics[width=0.95\linewidth]{../fig/placeholder}
		\caption{Bild rechts}
		\label{fig:right}
	\end{subfigure}
	\caption{Zwei Bilder nebeneinander}\label{fig:two_pictures}
\end{figure}\vspace{\skipAfterFigure pt}
{\footnotesize Quelle: \url{https://google.ch}, aufgerufen am 05.10.2019} 
\paragraph{Unterunterunterkapitel}
We have to go deeper.
\subsection{Weiteres Unterkapitel mit Tabelle}
\lipsum[1]
\begin{table}[H]\label{tab:table_description}
	\centering
	\caption{Tabellenbeschreibung}
	\begin{tabular}{|c|c|c|}
	\hline
  	\textbf{Methodik} & \textbf{Ermittelte Dehnung |\textit{$\epsilon$}|} & \textbf{Referenz}\\ \hline
  	a & \SI{720}{\micro\meter/\meter} & w \\ \hline
  	b & \SI{605.3}{\micro\meter/\meter} & x \\ \hline
  	c & \SI{655.77}{\micro\meter/\meter} & y \\ \hline
  	d & \SI{1247.10}{\micro\meter/\meter} & z \\ \hline
	\end{tabular}
\end{table}
\subsection{Weiteres Unterkapitel mit grosser Tabelle}
\begin{table}[H]
\centering
\caption{Eine etwas grössere Tabelle mit einer Spalte mit viel Text}\vspace{-3mm}
\begin{tabularx}{\textwidth}{|c|c|c|Y|}
\hline
\textbf{Modul/Funktion} & \textbf{Funktions-Detail} & \textbf{Status} & \textbf{Kommentar} \\ \hline
\multirow{4}{*}{\begin{tabular}[c]{@{}c@{}}Motorenkontroller \\ zur Ansteuerung \\ des DUT\end{tabular}} & Setzen digitaler Signale & OK & - \\ \cline{2-4} 
 & Einlesen digitaler Signale & OK & - \\ \cline{2-4} 
 & Ausgeben analoger Signale & NOK & Probleme mit DAC auf dem PCBA \\ \cline{2-4} 
 & Einlesen analoger Signale & OK & Abgleich der Nichtlinearität und Offset offen \\ \hline
\multirow{2}{*}{\begin{tabular}[c]{@{}c@{}}RHE zur Ansteuerung \\ der Hysteresebremse\end{tabular}} & Vorgabe Soll-Wert & NOK & Probleme mit DAC auf dem PCBA \\ \cline{2-4} 
 & Einlesen Ist-Wert & OK & Abgleich der Nichtlinearität und Offset offen \\ \hline
\begin{tabular}[c]{@{}c@{}}Druckmessung  \\ in Vakuumkammer\end{tabular} & Mit Varian FRG-700 & OK & Abgleich der Nichtlinearität und Offset offen \\ \hline
Temperaturmessung & Messung der Pt100 RTDs & OK & Genauigkeit verbesserungsfähig \\ \hline
Drehmomentmessung & Messen der DMS mittels WFB & N/A & Noch nicht getestet \\ \hline
\end{tabularx}
\end{table}
\textbf{Formel mit Nummerierung:}
\begin{equation}\label{equ:some_formula}
T = \dfrac{\sqrt{\left(A^2 \cdot R_0 - 4 \cdot B \cdot (-dR) \right) \cdot R_0} - A \cdot R_0}{2 \cdot B \cdot R_0} = \SI{0.0443}{\celsius}
\end{equation}
\textbf{Formel ohne Nummerierung:}\\
Elemente mit Labels können referenziert werden, wie beispielsweise die Formel \ref{equ:some_formula}.
\begin{equation*}
T = \dfrac{\sqrt{\left(A^2 \cdot R_0 - 4 \cdot B \cdot (-dR) \right) \cdot R_0} - A \cdot R_0}{2 \cdot B \cdot R_0} = \SI{0.0443}{\celsius}
\end{equation*}
\section{Zweites Hauptkapitel}
Beispiel für Pseudocode.\\
\begin{Verbatim}[tabsize=2,frame=single,label=Pseudo code of the firmware main-method,baselinestretch=0.8,numbers=left]
initialize all modules (of MC and peripherial)
allocate memory for test data buffers
seconds := 0 (timer-variable for logdata to be written to micro-sd-card)
DO ENDLESS:
	IF command received over over uart interface THEN
		execute command if it is valid
	END IF
	IF measurement due THEN
		toggle led to indicate start of measurement
		perform all measurements and create a string that contains all data
		IF values must be sent over uart THEN
			send all measurement values over uart
		END IF
		IF values must be written to micro sd card THEN
			write all values to the micro-sd-card
		END IF
	END IF	
	seconds := seconds + 1
REPEAT
\end{Verbatim}